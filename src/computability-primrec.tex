\subsection{原始再帰}

\newcommand{\syn}[1]{\mathtt{#1}}
\newcommand{\synZ}{\syn{z}}
\newcommand{\synS}{\syn{s}}
\newcommand{\synNat}{\syn{N}}


\begin{myDefinition}[形式的な自然数]
  記号$\synZ,\synS$による記号列が自然数であるとは,以下のように再帰的に定義される.
  \begin{enumerate}
    \item 記号列$\synZ$は自然数である.
    \item $n$が自然数ならば,記号列$\synS n$も自然数である.
  \end{enumerate}
\end{myDefinition}

\begin{myDefinition}[形式的な自然数の全ての集合]
  自然数である記号列を全て集めた集合を$\synNat$と書くことにする.
  すなわち,$\synNat \defeq \{ \synZ, \synS\synZ, \synS\synS\synZ,\dots\}$である.
\end{myDefinition}

\begin{myDefinition}[形式的な自然数の略記]
  自然数である記号列が$\synS$を$n$個含むとき,$\syn{n}$と略記する.
  例えば,以下のように略記する.
  \begin{enumerate}
    \item 記号列$\synZ$を$\syn{0}$と書くことにする.
    \item 記号列$\synS \syn{0}$を$\syn{1}$と書くことにする.すなわち$\syn{1}$とは$\synS \synZ$である.
    \item 記号列$\synS \syn{1}$を$\syn{2}$と書くことにする.すなわち$\syn{2}$とは$\synS \synS \synZ$である.
    \item 記号列$\synS \syn{2}$を$\syn{3}$と書くことにする.すなわち$\syn{3}$とは$\synS \synS \synS \synZ$である.
  \end{enumerate}
\end{myDefinition}

\newcommand{\cfun}[1]{\mathsf{#1}}
\begin{myDefinition}[関数]
  $n$個の形式的な自然数$\syn{x}_0,\syn{x}_1,\dots,\syn{x}_{n-1}$を入力として受け取り,
  1つの形式的な自然数$\syn{y}$を出力するシステムを,
  関数と呼び,$\cfun{f}:\synNat^n \to \synNat$と表す.

  関数$\cfun{f}:\synNat^n \to \synNat$に
  $\syn{x}_0,\dots,\syn{x}_{n-1}$を入力として与えると
  $\syn{y}$が出力されることを,
  $\cfun{f}(\syn{x}_0,\dots,\syn{x}_{n-1}) = \syn{y}$と書く.
  $\syn{x}_0,\dots,\syn{x}_{n-1}$が自明であるなら,
  $\cfun{f}(\vec{\syn{x}}) = \syn{y}$と略記する.
\end{myDefinition}

\newcommand{\cconst}[2]{\cfun{const}^{#1}_{\syn{#2}}}
\begin{myDefinition}[定数関数]
  $n$個の入力$\vec{\syn{x}}$を受け取るが,それらを全て破棄して
  定数$\syn{c}$を出力とする関数$\cconst{n}{c}:\synNat^n \to \synNat$を$n$変数定数関数と呼ぶ.
  すなわち,
  \begin{equation*}
    \cconst{n}{c}(\vec{\syn{x}}) \defeq \syn{c}
  \end{equation*}
  である.
\end{myDefinition}

\newcommand{\csucc}{\cfun{succ}}
\begin{myDefinition}[後者関数]
  次の関数$\csucc:\synNat \to \synNat$は後者関数と呼ぶ.
  \begin{equation*}
    \csucc(\syn{x}) \defeq \synS \syn{x}
  \end{equation*}
\end{myDefinition}


\newcommand{\cproj}[2]{\cfun{proj}^{#1}_{#2}}
\begin{myDefinition}[射影関数]
  $n$個の入力$\syn{x}_0,\dots,\syn{x}_{n-1}$を受け取り,そのうち指定した1つの入力$\syn{x}_i$を返す
  関数$\cproj{n}{i}:\synNat^n \to \synNat$を射影関数と呼ぶ.ただし,$0 \leq i \leq n$.
  すなわち,
  \begin{equation*}
    \cproj{n}{i}(\syn{x}_0,\dots,\syn{x}_{n-1}) \defeq \syn{x}_i
  \end{equation*}
  である.
\end{myDefinition}

\newcommand{\ccomp}[1]{\cfun{comp}_{#1}}
\begin{myDefinition}[関数合成]
  $m$個の関数$\cfun{f}_1,\dots,\cfun{f}_m:\synNat^n \to \synNat$と
  関数$g:\synNat^m \to \synNat$があるとき,
  関数$\ccomp{\cfun{f}_1,\dots,\cfun{f}_m,\cfun{g}}:\synNat^n \to \synNat$を以下のように定義する.
  \begin{align*}
    \ccomp{\cfun{f}_1,\dots,\cfun{f}_m,\cfun{g}}(\vec{\syn{x}}) \defeq \cfun{g}( \cfun{f}_1(\vec{\syn{x}}), \dots, \cfun{f}_m(\vec{\syn{x}}))
  \end{align*}
  $\ccomp{\cfun{f}_1,\dots,\cfun{f}_m,\cfun{g}}$は関数$\cfun{f}_1,\dots,\cfun{f}_m,\cfun{g}$の合成と呼ばれる.
\end{myDefinition}

\newcommand{\cprec}[1]{\cfun{prec}_{#1}}
\begin{myDefinition}[原始再帰]
  関数$\cfun{f}:\synNat^n \to \synNat$と
  関数$\cfun{f}:\synNat^{n+2} \to \synNat$があるとき,
  関数$\cprec{\cfun{f},\cfun{g}}:\synNat^{n+1} \to \synNat$を以下のように定義する.
  \begin{align*}
    \cprec{\cfun{f},\cfun{g}}(\vec{\syn{x}},\syn{0})           & \defeq \cfun{f}(\vec{\syn{x}})                                                                  \\
    \cprec{\cfun{f},\cfun{g}}(\vec{\syn{x}},\synS \syn{x}_{n}) & \defeq \cfun{g}(\vec{\syn{x}},\syn{x}_{n},\cprec{\cfun{f},\cfun{g}}(\vec{\syn{x}},\syn{x}_{n}))
  \end{align*}
  $\cprec{\cfun{f},\cfun{g}}$は関数$\cfun{f},\cfun{g}$の原始再帰と呼ばれる.
\end{myDefinition}

\begin{myDefinition}[原始再帰関数]
  原始再帰関数を以下のように定義する.
  \begin{enumerate}
    \item $n$変数定数関数$\cconst{n}{c}$は原始再帰関数である.
    \item 後者関数$\csucc$は原始再帰関数である.
    \item 射影関数$\cproj{n}{i}$は原始再帰関数である.
    \item 関数$\cfun{f}_1,\dots,\cfun{f}_m:\synNat^n \to \synNat$と関数$g:\synNat^m \to \synNat$が原始再帰関数であるとき,
          $\cfun{f}_1,\dots,\cfun{f}_n,\cfun{g}$の合成$\ccomp{\cfun{f}_1,\dots,\cfun{f}_m,\cfun{g}}:\synNat^n \to \synNat$は原始再帰関数.
    \item 関数$\cfun{f}:\synNat^n \to \synNat$と関数$g:\synNat^{n+2} \to \synNat$が原始再帰関数であるとき,
          $\cfun{f},\cfun{g}$の原始再帰$\cprec{\cfun{f},\cfun{g}}:\synNat^{n+1} \to \synNat$は原始再帰関数.
  \end{enumerate}
\end{myDefinition}

\newcommand{\czero}[1]{\cfun{zero}^{#1}}
\begin{myDefinition}[ゼロ関数$\czero{n}$]
  $\czero{n}:\synNat^n \to \synNat$を以下のように定義する.
  \begin{equation*}
    \czero{n} \defeq \cconst{n}{0}
  \end{equation*}
  すなわち,任意の入力$\vec{\syn{x}}$に対して$\czero{n}(\vec{\syn{x}}) = \syn{0}$である.
\end{myDefinition}

\newcommand{\cadd}{\cfun{add}}
\begin{myDefinition}[加算$\cadd$]
  $\cadd:\synNat^2 \to \synNat$を以下のように定義する.
  \begin{equation*}
    \cadd \defeq \cprec{\cproj{2}{0},\ccomp{\cproj{3}{2},\csucc}}
  \end{equation*}
  また,$\cadd(\syn{x}_0,\syn{x}_1)$を$\syn{x}_0 + \syn{x}_1$と略記する.
\end{myDefinition}

\begin{myRemark}
  定義に従って$\cadd$を簡単な形に展開すると以下のようになる.
  \begin{align*}
    \cadd(\syn{x}_0,\syn{0}) & \Rightarrow \cproj{2}{0}(\syn{x}_0,\syn{0}) \\
                             & \Rightarrow \syn{x}_0
  \end{align*}
  \begin{align*}
    \cadd\syn{x}_0,\synS \syn{x}_1) & \Rightarrow \csucc(\cproj{3}{2}(\syn{x}_0,\syn{x}_1,\cadd(\syn{x}_0,\syn{x}_1))) \\
                                    & \Rightarrow \csucc(\cadd(\syn{x}_0,\syn{x}_1))
  \end{align*}
\end{myRemark}

\newcommand{\cmul}{\cfun{mul}}
\begin{myDefinition}[乗算$\cmul$]
  $\cmul:\synNat^2 \to \synNat$を以下のように定義する.
  \begin{equation*}
    \cmul \defeq \cprec{\czero{2},\ccomp{\cproj{3}{0},\cproj{3}{2},\cadd}}
  \end{equation*}
  また,$\cmul(\syn{x}_0,\syn{x}_1)$を$\syn{x}_0 \times \syn{x}_1$と略記する.
\end{myDefinition}

\begin{myRemark}
  定義に従って$\cmul$を簡単な形に展開すると以下のようになる.
  \begin{align*}
    \cmul(\syn{x}_0,\syn{0}) & \Rightarrow \czero{2}(\syn{x}_0,\syn{0}) \\
                             & \Rightarrow \syn{0}
  \end{align*}
  \begin{align*}
    \cmul(\syn{x}_0,\synS \syn{x}_1) & \Rightarrow \cadd(\cproj{3}{0}(\syn{x}_0,\syn{x}_1,\cmul(\syn{x}_0,\syn{x}_1)),\cproj{3}{2}(\syn{x}_0,\syn{x}_1,\cmul(\syn{x}_0,\syn{x}_1))) \\
                                     & \Rightarrow \cadd(\syn{x}_0,\cmul(\syn{x}_0,\syn{x}_1))
  \end{align*}
\end{myRemark}

\newcommand{\cpred}{\cfun{pred}}
\begin{myDefinition}[前者減算$\cpred$]
  $\cpred:\synNat^2 \to \synNat$を以下のように定義する.
  \begin{equation*}
    \cpred \defeq \cprec{\czero{2},\cproj{3}{1}}
  \end{equation*}
\end{myDefinition}

\newcommand{\cmsub}{\cfun{msub}}
\begin{myDefinition}[補正付き減算$\cmsub$]
  $\cmsub:\synNat^2 \to \synNat$を以下のように定義する.
  \begin{equation*}
    \cmsub \defeq \cprec{\cproj{2}{0},\ccomp{\cproj{3}{0},\cpred}}
  \end{equation*}
  また,$\cmsub(\syn{x}_0,\syn{x}_1)$を$\syn{x}_0 \ominus \syn{x}_1$と略記する.
\end{myDefinition}

\newcommand{\cispos}{\cfun{pos?}}
\begin{myDefinition}[正数判定$\cispos$]
  $\cispos:\synNat \to \synNat$を以下のように定義する.
  \begin{equation*}
    \cispos \defeq \cprec{\czero{1},\cconst{2}{1}}
  \end{equation*}
\end{myDefinition}

\newcommand{\ciszero}{\cfun{zero?}}
\begin{myDefinition}[ゼロ判定$\ciszero$]
  $\ciszero:\synNat \to \synNat$を以下のように定義する.
  \begin{equation*}
    \ciszero \defeq \cprec{\cconst{1}{1},\czero{2}}
  \end{equation*}
\end{myDefinition}

\newcommand{\cswap}[1]{\cfun{swap}_{\cfun{#1}}}
\begin{myDefinition}[引数取り替え$\cswap{f}$]
  $\cfun{f}:\synNat^2 \to \synNat$に対し,
  $\cswap{f}:\synNat^2 \to \synNat$を以下のように定義する.
  \begin{equation*}
    \cswap{f} \defeq \ccomp{\cproj{2}{1},\cproj{2}{0},\cfun{f}}
  \end{equation*}
\end{myDefinition}

\begin{myRemark}
  \label{remark:comp-swap}
  例えば,$\cadd$に対しての引数取り換え$\cswap{\cadd}$を展開すると,
  \begin{align*}
    \cswap{\cadd}(\syn{x}_0,\syn{x}_1)
     & \Rightarrow \ccomp{\cproj{2}{1},\cproj{2}{0},\cadd}(\syn{x}_0,\syn{x}_1)               \\
     & \Rightarrow \cadd(\cproj{2}{1}(\syn{x}_0,\syn{x}_1),\cproj{2}{0}(\syn{x}_0,\syn{x}_1)) \\
     & \Rightarrow \cadd(\syn{x}_1,\syn{x}_0)                                                 \\
  \end{align*}
  となる.
  この関数により,任意の2項演算,例えば加算$\cadd$,乗算$\cmul$,補正付き減算$\cmsub$の引数
  を取り替えた関数を定義することができる.
\end{myRemark}

\begin{myTheorem}
  加算$\cadd$,乗算$\cmul$,
  前者関数$\cpred$,
  補正付き減算$\cmsub$,
  正数判定$\cispos$,ゼロ判定$\cispos$,
  引数取り替え$\cswap{f}$
  は原始再帰関数である.
\end{myTheorem}

\begin{myDefinition}[特性関数]
  関係$\syn{R} \subseteq \synNat^n$について,
  $\chi_{\syn{R}}:\synNat^n \to \synNat$が以下を満たすとき,
  $\chi_{\syn{R}}$を$\syn{R}$の特性関数と呼ぶ.
  \begin{equation*}
    \chi_{\syn{R}}(\vec{\syn{x}}) \defeq
    \begin{cases}
      \syn{1} & \text{if } \vec{\syn{x}} \in \syn{R} \\
      \syn{0} & \text{otherwise}
    \end{cases}
  \end{equation*}
\end{myDefinition}

\begin{myDefinition}[原始再帰的関係]
  関係$\syn{R} \subseteq \synNat^n$の特性関数$\chi_{\syn{R}}$が原始再帰関数であるとき,
  $\syn{R}$は原始再帰的な関係であるという.
  特に$n=1$のとき,$\syn{R} \subseteq \synNat$は原始再帰的な集合であるという.
\end{myDefinition}

\begin{myTheorem}[等値関係]
  \label{thm:eq-prim-rec}
  「自然数$\syn{x}_0$と$\syn{x}_1$が同じ個数の$\synS$を持っている」
  という関係$\syn{Eq} \subseteq \synNat^2$は原始再帰的である.
  以下,$(\syn{x}_0,\syn{x}_1) \in \syn{Eq}$を$\syn{x}_0 = \syn{x}_1$と表す.
  $\syn{x}_0 = \syn{x}_1$は等値関係と呼ぶ.
\end{myTheorem}
\begin{myProof}[\ref{thm:eq-prim-rec}]
  $\syn{Eq}$の特性関数$\chi_{\syn{Eq}}$を以下のように定義すると要件を満たし,原始再帰的関数である.
  \begin{equation*}
    \chi_{\syn{Eq}}(\syn{x}_0,\syn{x}_1) \defeq
    \ciszero((\syn{x}_0 \ominus \syn{x}_1) + (\syn{x}_1 \ominus \syn{x}_0))
  \end{equation*}
  ここで,$\syn{x}_1 \ominus \syn{x}_0$は$\cswap{\cmsub}$から構成されることを\ref*{remark:comp-swap}で注意した.
\end{myProof}

\begin{myTheorem}[大小関係]
  \label{thm:lt-prim-rec}
  「自然数$\syn{x}_1$は$\syn{x}_0$より多くの個数の$\synS$を持っている」
  あるいは
  「自然数$\syn{x}_0$は$\syn{x}_1$より少ない個数の$\synS$を持っている」
  という関係$\syn{Lt} \subseteq \synNat^2$は原始再帰的である.
  以下,$(\syn{x}_0,\syn{x}_1) \in \syn{Lt}$を$\syn{x}_0 < \syn{x}_1$と表す.
  $\syn{x}_0 < \syn{x}_1$は大小関係と呼ぶ.
\end{myTheorem}
\begin{myProof}[\ref{thm:eq-prim-rec}]
  $\syn{Lt}$の特性関数$\chi_{\syn{Lt}}$を以下のように定義すると要件を満たし,原始再帰的関数である.
  \begin{equation*}
    \chi_{\syn{Lt}}(\syn{x}_0,\syn{x}_1) \defeq
    \cispos(\syn{x}_1 \ominus \syn{x}_0)
  \end{equation*}
\end{myProof}

\begin{myDefinition}[論理演算]
  関係$\syn{R},\syn{S} \subseteq \synNat^n$について,論理演算$\lnot\syn{R},\syn{R}\land\syn{S},\syn{R}\lor\syn{S}$を以下のように定義する.
  \begin{enumerate}
    \item $\vec{\syn{x}} \in \lnot \syn{R} \iff \vec{\syn{x}} \notin \lnot \syn{R}$
    \item $\vec{\syn{x}} \in \syn{R} \land \syn{S} \iff \text{$\vec{\syn{x}} \in \syn{R}$と$\vec{\syn{x}} \in \syn{S}$の両方が成立}$
    \item $\vec{\syn{x}} \in \syn{R} \lor \syn{S} \iff \text{$\vec{\syn{x}} \in \syn{R}$または$\vec{\syn{x}} \in \syn{S}$の少なくとも一方が成立}$
  \end{enumerate}
\end{myDefinition}

\begin{myTheorem}
  \label{thm:prim-rec-logical}
  関係$\syn{R},\syn{S} \subseteq \synNat^n$が原始再帰関係であるとき,$\lnot\syn{R},\syn{R}\land\syn{S},\syn{R}\lor\syn{S}$は原始再帰関係である.
\end{myTheorem}
\begin{myProof}[\ref{thm:prim-rec-logical}]
  原始再帰関係$\syn{R},\syn{S}$の特性関数をそれぞれ$\chi_{\syn{R}},\chi_{\syn{S}}$とする.
  定義より,$\chi_{\syn{R}},\chi_{\syn{S}}$は原始再帰関数である.
  \begin{enumerate}
    \item $\lnot\syn{R}$の特性関数$\chi_{\lnot\syn{R}}$を
          $\chi_{\lnot\syn{R}}(\vec{\syn{x}}) \defeq \ciszero(\chi_{\syn{R}}(\vec{\syn{x}}))$
          と定義すると要件を満たし,原始再帰的関数である.
    \item $\syn{R}\land\syn{S}$の特性関数$\chi_{\syn{R}\land\syn{S}}$を
          $\chi_{\syn{R}\land\syn{S}}(\vec{\syn{x}}) \defeq \chi_{\syn{R}}(\vec{\syn{x}}) \times \chi_{\syn{S}}(\vec{\syn{x}})$
          と定義すると要件を満たし,原始再帰的関数である.
    \item $\syn{R}\lor\syn{S}$の特性関数$\chi_{\syn{R}\lor\syn{S}}$を
          $\chi_{\syn{R}\lor\syn{S}}(\vec{\syn{x}}) \defeq \cispos(\chi_{\syn{R}}(\vec{\syn{x}}) + \chi_{\syn{S}}(\vec{\syn{x}}))$
          と定義すると要件を満たし,原始再帰的関数である.
  \end{enumerate}
\end{myProof}

\begin{myDefinition}
  関係$\lnot \syn{Eq}$を$\syn{NEq}$と定義する.
  以下,$(\syn{x}_0,\syn{x}_1) \in \syn{Neq}$を$\syn{x}_0 \neq \syn{x}_1$と表す.
\end{myDefinition}

\begin{myCorollary}
  関係$\syn{Neq}$は原始再帰関係である.
\end{myCorollary}

\begin{myDefinition}
  関係$\syn{Eq} \lor \syn{Lt}$を$\syn{LEq}$と定義する.
  以下,$(\syn{x}_0,\syn{x}_1) \in \syn{LEq}$を$\syn{x}_0 \leq \syn{x}_1$と表す.
\end{myDefinition}

\begin{myCorollary}
  関係$\syn{LEq}$は原始再帰関係である.
\end{myCorollary}

\begin{myDefinition}[有界総和]
  関数$\syn{f}:\synNat^{n+1} \to \synNat$に対し,関数$\syn{\Sigma_f}:\synNat^{n+1} \to \synNat$を以下のように定める.
  \begin{equation*}
    \syn{\Sigma_f}(\vec{\syn{x}},\syn{0}) \defeq \syn{f}(\vec{\syn{x}},\syn{0})
  \end{equation*}
  \begin{equation*}
    \syn{\Sigma_f}(\vec{\syn{x}},\synS\syn{x}_n) \defeq \syn{\Sigma_f}(\vec{\syn{x}},\syn{x}_n) + \syn{f}(\vec{\syn{x}},\synS\syn{x}_n)
  \end{equation*}
  $\syn{\Sigma_f}(\vec{\syn{x}},\syn{x}_n)$を,$\syn{f}$の$\vec{\syn{x}}$に対しての$\syn{x}_n$までの総和と呼ぶ.
\end{myDefinition}

\begin{myDefinition}[有界総乗]
  関数$\syn{f}:\synNat^{n+1} \to \synNat$に対し,関数$\syn{\Pi_f}:\synNat^{n+1} \to \synNat$を以下のように定める.
  \begin{equation*}
    \syn{\Pi_f}(\vec{\syn{x}},\syn{0}) \defeq \syn{f}(\vec{\syn{x}},\syn{0})
  \end{equation*}
  \begin{equation*}
    \syn{\Pi_f}(\vec{\syn{x}},\synS\syn{x}_n) \defeq \syn{\Pi_f}(\vec{\syn{x}},\syn{x}_n) \times \syn{f}(\vec{\syn{x}},\synS\syn{x}_n)
  \end{equation*}
  $\syn{\Pi_f}(\vec{\syn{x}},\syn{x}_n)$を,$\syn{f}$の$\vec{\syn{x}}$に対しての$\syn{x}_n$までの総乗と呼ぶ.
\end{myDefinition}

\begin{myDefinition}[有界全称量化]
  関係$\syn{R} \subseteq \synNat^{n+1}$について,
  関係$\forall_{\syn{y} \leq \syn{x}_n} \syn{R}(\vec{\syn{x}},\syn{y}) \subseteq \synNat^{n+1}$
  を以下のように定義する.
  \begin{equation*}
    (\vec{\syn{x}},\syn{x}_n) \in \forall_{\syn{y} \leq \syn{x}_n} \syn{R}(\vec{\syn{x}},\syn{y})
    \iff
    \text{$\syn{x}_n$以下の任意の$\syn{y} \in \synNat$で$(\vec{\syn{x}},\syn{y}) \in \syn{R}$となる}
  \end{equation*}
\end{myDefinition}

\begin{myTheorem}
  関係$\syn{R} \subseteq \synNat^{n+1}$が原始再帰関係であるとき,
  関係$\forall_{\syn{y} \leq \syn{x}_n} \syn{R}(\vec{\syn{x}},\syn{y})$
  も原始再帰関係.
\end{myTheorem}
\begin{myProof}
  以下の関数$\chi_{\forall_{\syn{y} \leq \syn{x}_n} \syn{R}(\vec{\syn{x}},\syn{y})}$
  は原始再帰関数である.
  \begin{equation*}
    \chi_{\forall_{\syn{y} \leq \syn{x}_n} \syn{R}(\vec{\syn{x}},\syn{y})}
    (\vec{\syn{x}},\syn{x}_n)
    \defeq
    \syn{\Pi}_{\chi_{\cfun{R}}}(\vec{\syn{x}},\syn{x}_n)
  \end{equation*}
  展開すると,
  \begin{align*}
    \syn{\Pi}_{\chi_{\cfun{R}}}(\vec{\syn{x}},\syn{x}_n)
     & \Rightarrow
    \chi_{\cfun{R}}(\vec{\syn{x}},\syn{0}) \times \chi_{\cfun{R}}(\vec{\syn{x}},\syn{1}) \times \cdots \times \chi_{\cfun{R}}(\vec{\syn{x}},\syn{x}_n)
  \end{align*}
  となり,$\forall_{\syn{y} \leq \syn{x}_n} \syn{R}(\vec{\syn{x}},\syn{y})$の特性関数としての
  要件を満たしていることは明らか.
\end{myProof}

\begin{myDefinition}[有界存在量化]
  関係$\syn{R} \subseteq \synNat^{n+1}$について,
  関係$\exists_{\syn{y} \leq \syn{x}_n} \syn{R}(\vec{\syn{x}},\syn{y}) \subseteq \synNat^{n+1}$
  を以下のように定義する.
  \begin{equation*}
    (\vec{\syn{x}},\syn{x}_n) \in \exists_{\syn{y} \leq \syn{x}_n} \syn{R}(\vec{\syn{x}},\syn{y})
    \iff
    \text{$\syn{x}_n$以下のある$\syn{y} \in \synNat$で$(\vec{\syn{x}},\syn{y}) \in \syn{R}$となる$\syn{y}$が存在する}
  \end{equation*}
\end{myDefinition}

\begin{myTheorem}
  関係$\syn{R} \subseteq \synNat^{n+1}$が原始再帰関係であるとき,
  関係$\exists_{\syn{y} \leq \syn{x}_n} \syn{R}(\vec{\syn{x}},\syn{y})$
  も原始再帰関係.
\end{myTheorem}
\begin{myProof}
  以下の関数$\chi_{\exists_{\syn{y} \leq \syn{x}_n} \syn{R}(\vec{\syn{x}},\syn{y})}$
  は原始再帰関数である.
  \begin{equation*}
    \chi_{\exists_{\syn{y} \leq \syn{x}_n} \syn{R}(\vec{\syn{x}},\syn{y})}
    (\vec{\syn{x}},\syn{x}_n)
    \defeq
    \cispos(\syn{\Sigma}_{\chi_{\cfun{R}}}(\vec{\syn{x}},\syn{x}_n))
  \end{equation*}
  展開すると,
  \begin{align*}
    \cispos(\syn{\Sigma}_{\chi_{\cfun{R}}}(\vec{\syn{x}},\syn{x}_n))
     & \Rightarrow
    \cispos(\chi_{\cfun{R}}(\vec{\syn{x}},\syn{0}) + \chi_{\cfun{R}}(\vec{\syn{x}},\syn{1}) + \cdots + \chi_{\cfun{R}}(\vec{\syn{x}},\syn{x}_n))
  \end{align*}
  となり,$\exists_{\syn{y} \leq \syn{x}_n} \syn{R}(\vec{\syn{x}},\syn{y})$の特性関数としての
  要件を満たしていることは明らか.
\end{myProof}

\begin{myTheorem}[偶数の集合]
  0または偶数個の$\synS$を持っている自然数の集合$\syn{Even} \subseteq \synNat$,
  すなわち,$\syn{Even} \defeq \{\syn{0}, \syn{2}, \syn{4},\dots\}$は原始再帰集合である.
\end{myTheorem}

\begin{myTheorem}[約数関係]
  「$\syn{x}_0$は$\syn{x}_1$を割り切る」を意味する$\syn{div} \subseteq \synNat^2$は原始再帰関係である.
\end{myTheorem}

\begin{myTheorem}[素数の集合]
  素数個の$\synS$を持っている自然数の集合$\syn{Prime} \subseteq \synNat$,
  すなわち,$\syn{Prime} \defeq \{\syn{2}, \syn{3}, \syn{5},\dots\}$は原始再帰集合である.
\end{myTheorem}

\begin{myTheorem}[素数の探索範囲の上界]
  $p_n$を$n$番目の素数としたとき,$n+1$番目の素数$p_{n+1}$は$p_n! + 1$以下に存在する.
\end{myTheorem}

\begin{myDefinition}[形式的の素数]
  $n+1$番目の素数を$p_n$とする.$\syn{p_n}$は$p_n$個の$\synS$を持つ自然数を表すと定義する.
  例えば,$\syn{p_0} = \syn{2}, \syn{p_1} = \syn{3}, \syn{p_2} = \syn{5}$である.
\end{myDefinition}

\begin{myDefinition}
  $\syn{x}$に対して$\syn{p_x}$を計算する関数を$\cfun{pr}:\synNat \to \synNat$とする.
\end{myDefinition}

\begin{myTheorem}
  $\cfun{pr}$は原始再帰関数である.
\end{myTheorem}
