\subsection{意味論}

\begin{myDefinition}[命題記号の解釈と真理値]
  \label{def:prop:interpret-props}
  命題記号の解釈と真理値を以下のように定める.
  \begin{enumerate}
    \item 命題記号$p_n$を,正しいか誤っているかのどちらかを判断できるような命題へと解釈する写像を,単に解釈と読んで,$\PropInterpret$で表す.
    \item $\PropInterpret$で解釈された命題を,$\PropInterpret(p_n)$として表す.
    \item 命題$\PropInterpret(p_n)$が正しいとき,命題は真と呼び,$\PropInterpret(p_n) = \True$と表す.
    \item 命題$\PropInterpret(p_n)$が誤っているとき,命題は偽と呼び,$\PropInterpret(p_n) = \False$と表す.
  \end{enumerate}
\end{myDefinition}

\begin{myExample}
  \label{def:prop:interpret-weather}
  解釈$\PropInterpret$は,命題記号$p_0$を「2023年6月3日の東京では雨が降った」という文に解釈するとする.
  すなわち,$\PropInterpret(p_0) =$である.

  このとき,この命題の真偽は,実際に2023年6月3日の東京では雨が降ったかどうかを確かめることで判断できる.
  この命題は真である\footnote{text}
  ため,$\PropInterpret(p_0) = \True$である.
\end{myExample}

\begin{myExample}
  \label{def:prop:leapyear}
  \ref{def:prop:interpret-weather}とは別の解釈$\PropInterpret'$では,命題記号$p_0$を「2023年はうるう年である」という文に解釈するとする.
  すなわち,$\PropInterpret'(p_0) =$である.

  このとき,この命題の真偽は,うるう年の定義を調べることで判断できる.
  うるう年は,西暦年が「4の倍数であり,かつ,100の倍数でない」または「400の倍数である」であるときである.
  2023は4の倍数でもなければ,400の倍数でもないので,2023年はうるう年ではない.
  すなわち,この命題は偽であり,$\PropInterpret'(p_0) = \False$である.
\end{myExample}

\begin{myDefinition}[付値関数]
  \label{def:prop:valuation}
  論理式についての真理値を以下のように定める.
  \begin{enumerate}
    \item ある解釈$\PropInterpret$の上で命題論理の論理式に$\True$または$\False$を割り当てる写像を,
          $\PropInterpret$の付値と呼び,$\PropValuation_\PropInterpret$で表す.
    \item 付値$\PropValuation_\PropInterpret$によって論理式$\varphi$に割り当てられる$\True,\False$を,
          $\PropValuation_\PropInterpret$での$\varphi$の真理値と呼ぶ.
          % \item ある付値$\PropValuation_\PropInterpret$が,
          %       論理式$\varphi$に真を割り当てるとき,$\PropValuation_\PropInterpret(\varphi) = \True$と書く.
          % \item ある付値$\PropValuation_\PropInterpret$が,
          %       論理式$\varphi$に偽を割り当てるとき,$\PropValuation_\PropInterpret(\varphi) = \False$と書く.
    \item 付値$\PropValuation_\PropInterpret$は論理式の定義\ref*{def:prop:formulae}に従って,再帰的に定義される.
          \begin{enumerate}
            \item $\PropValuation_\PropInterpret(p_n) = \PropInterpret(p)$と定義する.
            \item $\varphi$は論理式とする.
                  $\PropValuation_\PropInterpret(\varphi) = \True$のとき,そのときに限り,
                  $\PropValuation_\PropInterpret(\neg \varphi) = \False$と定義する.
                  そうでないとき,
                  $\PropValuation_\PropInterpret(\neg \varphi) = \True$と定義する.
            \item $\varphi,\psi$は論理式とする.
                  $\PropValuation_\PropInterpret(\varphi) = \True$かつ$\PropValuation_\PropInterpret(\psi) = \False$のとき,そのときに限り,
                  $\PropValuation_\PropInterpret(\varphi \limp \psi) = \False$と定義する.
                  それ以外のとき,
                  $\PropValuation_\PropInterpret(\varphi \limp \psi) = \True$と定義する.
          \end{enumerate}
  \end{enumerate}
\end{myDefinition}

\begin{myRemark}[真理値表]
  $\varphi, \psi$は論理式とし,$\PropValuation_\PropInterpret$は付値とする.
  $\lnot \varphi$と$\varphi \to \psi$の$\PropValuation_\PropInterpret$の真偽を割り当てを,
  表にしてまとめてみると,以下のようになる.
  \begin{table}
    \begin{tabular}{|c|c|}\hline
      $\PropValuation_\PropInterpret(\varphi)$ & $\PropValuation_\PropInterpret(\lnot\varphi)$ \\ \hline
      $\True$                                  & $\False$                                      \\ \hline
      $\False$                                 & $\True$                                       \\ \hline
    \end{tabular}
  \end{table}
  \begin{table}
    \begin{tabular}{|c|c|c|}\hline
      $\PropValuation_\PropInterpret(\varphi)$ & $\PropValuation_\PropInterpret(\psi)$ & $\PropValuation_\PropInterpret(\varphi \limp \psi)$ \\ \hline
      $\True$                                  & $\True$                               & $\True$                                             \\ \hline
      $\True$                                  & $\False$                              & $\False$                                            \\ \hline
      $\False$                                 & $\True$                               & $\True$                                             \\ \hline
      $\False$                                 & $\False$                              & $\True$                                             \\ \hline
    \end{tabular}
  \end{table}
\end{myRemark}

\begin{myRemark}[構文論的同値と真理値]
  \label{rmk:prop:syntax-equiv-truth}
  $\varphi, \psi$は論理式とし,$\PropValuation_\PropInterpret$は付値とする.
  $\varphi,\psi$が構文論的に同値であるなら,
  $\PropValuation_\PropInterpret$での真理値も一致する.
  すなわち,$\varphi \equiv \psi$ならば
  $\PropValuation_\PropInterpret(\varphi) = \PropValuation_\PropInterpret(\psi)$である.
  逆に,真理値が一致しているからといって,構文論的に同値であるとは限らない.
  すなわち,$\PropValuation_\PropInterpret(\varphi) = \PropValuation_\PropInterpret(\psi)$ならば
  $\varphi \equiv \psi$であるとは限らない.
\end{myRemark}

\begin{myCorollary}[省略記号$\land, \lor, \liff$の真理値]
  \label{cor:prop:valuation-sugar-syntax}
  $\varphi, \psi$は論理式とし,$\PropValuation_\PropInterpret$は付値とする.
  \ref{def:prop:sugar-sytnax}で導入した省略記号$\land, \lor, \liff$を展開した論理式は,
  \ref{exp:prop:syntax-equiv}で示した.
  \ref*{rmk:prop:syntax-equiv-truth}より,展開前と展開後の論理式の真理値は一致しなければならない.
  すなわち,次が成り立つ.
  \begin{enumerate}
    \item $\PropValuation_\PropInterpret(\varphi \lor \psi) = \PropValuation_\PropInterpret(\lnot \varphi \limp \psi)$
    \item $\PropValuation_\PropInterpret(\varphi \land \psi) = \PropValuation_\PropInterpret(\lnot(\lnot\varphi \lor \lnot\psi))$
    \item $\PropValuation_\PropInterpret(\varphi \liff \psi) = \PropValuation_\PropInterpret((\varphi \limp \psi) \land (\psi \limp \varphi))$
  \end{enumerate}
  これを,真理値表にまとめると以下のようになる.
  \begin{table}
    \begin{tabular}{|c|c|c|c|}\hline
      $\PropValuation_\PropInterpret(\varphi)$ & $\PropValuation_\PropInterpret(\psi)$ & $\PropValuation_\PropInterpret(\lnot\varphi)$ & $\PropValuation_\PropInterpret(\varphi \lor \psi) = \PropValuation_\PropInterpret(\lnot \varphi \limp \psi)$ \\ \hline
      $\True$                                  & $\True$                               & $\False$                                      & $\True$                                                                                                      \\ \hline
      $\True$                                  & $\False$                              & $\False$                                      & $\True$                                                                                                      \\ \hline
      $\False$                                 & $\True$                               & $\True$                                       & $\True$                                                                                                      \\ \hline
      $\False$                                 & $\False$                              & $\True$                                       & $\False$                                                                                                     \\ \hline
    \end{tabular}
  \end{table}

  \begin{table}
    \begin{tabular}{|c|c|c|c|c|c|}\hline
      $\PropValuation_\PropInterpret(\varphi)$ & $\PropValuation_\PropInterpret(\psi)$ & $\PropValuation_\PropInterpret(\lnot\varphi)$ & $\PropValuation_\PropInterpret(\lnot\psi)$ & $\PropValuation_\PropInterpret(\lnot \varphi \lor \lnot\psi)$ & $\PropValuation_\PropInterpret(\varphi \land \psi) = \PropValuation_\PropInterpret(\lnot(\lnot\varphi \lor \lnot\psi))$ \\ \hline
      $\True$                                  & $\True$                               & $\False$                                      & $\False$                                   & $\False$                                                      & $\True$                                                                                                                 \\ \hline
      $\True$                                  & $\False$                              & $\False$                                      & $\True$                                    & $\True$                                                       & $\False$                                                                                                                \\ \hline
      $\False$                                 & $\True$                               & $\True$                                       & $\False$                                   & $\True$                                                       & $\False$                                                                                                                \\ \hline
      $\False$                                 & $\False$                              & $\True$                                       & $\True$                                    & $\True$                                                       & $\False$                                                                                                                \\ \hline
    \end{tabular}
  \end{table}

  \begin{table}
    \begin{tabular}{|c|c|c|c|c|}\hline
      $\PropValuation_\PropInterpret(\varphi)$ & $\PropValuation_\PropInterpret(\psi)$ & $\PropValuation_\PropInterpret(\varphi \to \psi)$ & $\PropValuation_\PropInterpret(\psi \to \varphi)$ & $\PropValuation_\PropInterpret(\varphi \liff \psi) = \PropValuation_\PropInterpret((\varphi \limp \psi) \land (\psi \limp \varphi))$ \\ \hline
      $\True$                                  & $\True$                               & $\True$                                           & $\True$                                           & $\True$                                                                                                                              \\ \hline
      $\True$                                  & $\False$                              & $\False$                                          & $\True$                                           & $\False$                                                                                                                             \\ \hline
      $\False$                                 & $\True$                               & $\True$                                           & $\False$                                          & $\False$                                                                                                                             \\ \hline
      $\False$                                 & $\False$                              & $\True$                                           & $\True$                                           & $\True$                                                                                                                              \\ \hline
    \end{tabular}
  \end{table}
\end{myCorollary}
